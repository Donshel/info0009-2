\documentclass[a4paper, 12pt]{article}

\def\languages{english, french}

%%%%%%%%%%%%%%%%%%% Libraries

\newgeometry{margin = 2.5cm}
\makeatletter
\begin{titlepage}
	\begin{minipage}[t][0.425\textheight][t]{\textwidth}
		\begin{center}
		    \ifx\toptitle\undefined
    		    \vfill
    		    \ifx\logopath\undefined
    		    \else
    			    \includegraphics[height=0.2125\textheight]{\logopath}
    			\fi
    		\else
    		    \ifx\logopath\undefined
    		    \else
    			    \includegraphics[height=0.15\textheight]{\logopath}
    			\fi
    			\vfill
    			{\huge \textsc{\toptitle}}
			\fi
			\vfill
		\end{center}
	\end{minipage}
	\vfill
	\begin{minipage}{\textwidth}
		\hspace{0.5em}
		\begin{mdframed}[linewidth = 2pt, innertopmargin = 1em, innerbottommargin = 1em, leftline = false, rightline = false]
			\begin{center}
				{\huge \bfseries \@title}
			\end{center}
		\end{mdframed}
		\hspace{0.5em}
	\end{minipage}
	\vfill
	\begin{minipage}[b][0.425\textheight][t]{\textwidth}
			\ifx\subtitle\undefined
			\else
			    \vspace{-0.5em}
			    \begin{center}
				    {\LARGE \subtitle}
				\end{center}
			\fi
			\vfill
			\ifx\rightauthor\undefined
			    \begin{center}
			        \ifx\authorhead\undefined
			        \else
		                {\large\it \authorhead\\[0.5em]}
		            \fi
			        {\large \@author}
			    \end{center}
			\else
			    \begin{minipage}[t]{0.5\textwidth}
			        \begin{flushleft}
			            \ifx\authorhead\undefined
			            \else
			                {\large\it \authorhead\\[0.5em]}
			            \fi
				        {\large \@author}
				    \end{flushleft}
				\end{minipage}
				\begin{minipage}[t]{0.5\textwidth}
				    \begin{flushright}
				        \ifx\rightauthorhead\undefined
			            \else
			                {\large\it \rightauthorhead\\[0.5em]}
			            \fi
				        {\large \rightauthor}
				    \end{flushright}
				\end{minipage}
			\fi
			\vfill
			\begin{center}
			    \ifx\context\undefined
			    \else
			        {\large \context \\[0.5em]}
			    \fi
			    {\large \@date}
			\end{center}
	\end{minipage}
\end{titlepage}
\makeatother
\restoregeometry
%%%%%%%%%% Packages

\usepackage{amsmath}
\usepackage{amssymb}
\usepackage{bm}
\usepackage{esint}
\usepackage[makeroom]{cancel}

%%%%%%%%%% Features

%%%%% Macros

\newcommand{\rbk}[1]{\left(#1\right)}
\newcommand{\cbk}[1]{\left\{#1\right\}}
\newcommand{\sbk}[1]{\left[#1\right]}
\newcommand{\abs}[1]{\left|#1\right|}
\newcommand{\norm}[1]{\left\|#1\right\|}

\newcommand{\fact}[1]{#1!}
\newcommand{\e}[1]{\mathbf{e}_{#1}}
\newcommand{\deriv}{\mathrm{d}}
\DeclareMathOperator{\tr}{tr}

\def\Rl{\mathbb{R}}
\def\Cx{\mathbb{C}}
\def\Na{\mathbb{N}}
\def\Zi{\mathbb{Z}}

%%%%%%%%%% Packages

\usepackage{siunitx}

%%%%%%%%%% Features

%%%%% Settings

\ifx\decimalsign\undefined
\else
    \sisetup{output-decimal-marker = \decimalsign}
\fi

%%%%%%%%%% Packages

\usepackage{inconsolata}
\usepackage{listings}

%%%%%%%%%% Features

%%%%% Commands

\newcommand{\Nstyle}[1]{
    \lstdefinestyle{N#1}{
        style=#1,
        %%%%%
        numbers=left
    }
}

\newcommand{\tbFstyle}[1]{
    \lstdefinestyle{tbF#1}{
        style=#1,
        %%%%%
        aboveskip=1.5em,
        frame=tb,
    }
}

\newcommand{\Fstyle}[1]{
    \lstdefinestyle{F#1}{
        style=#1,
        %%%%%
        aboveskip=1.5em,
        frame=single,
        framesep=0em,
        rulesep=0em,
        xleftmargin=0.75em,
        xrightmargin=0.75em,
        framexleftmargin=0.75em,
        framexrightmargin=0.75em,
        framextopmargin=0.5em,
        framexbottommargin=0.5em,
        %%%%%
        numbersep=1.25em
    }
}

\newcommand{\NtbFstyle}[1]{
    \tbFstyle{#1}
    \Nstyle{tbF#1}
}

\newcommand{\NFstyle}[1]{
    \Fstyle{#1}
    \lstdefinestyle{NF#1}{
        style=f#1,
        %%%%%
        xleftmargin=2.75em,
        framexleftmargin=2.75em,
        %%%%%
        numbers=left,
        numbersep=1em
    }
}

%%%%% Styles

\lstdefinestyle{default}{
    aboveskip=1em,
    breaklines=true,
    breakatwhitespace=true,
    columns=fixed,
	extendedchars=true,
    upquote=true,
	tabsize=4,
    %%%%%
    framerule=0.66pt,
    captionpos=b,
	%%%%%
    basicstyle=\footnotesize\ttfamily,
    numberstyle=\footnotesize\ttfamily,
    showstringspaces=false
}
\Nstyle{default}
\NFstyle{default}
\NtbFstyle{default}

\lstdefinestyle{monokai}{
    style=Fdefault,
    %%%%%
    backgroundcolor=\color[HTML]{272822},
    framerule=0em,
    %%%%%
    basicstyle=\footnotesize\ttfamily\color[HTML]{f8f8f2},
    numberstyle=\footnotesize\ttfamily\color[HTML]{272822},
    commentstyle=\color[HTML]{75715e},
    keywordstyle=[1]{\color[HTML]{f92672}},
    keywordstyle=[2]{\color[HTML]{A6E22E}},
    keywordstyle=[3]{\color[HTML]{ae81ff}},
    stringstyle=\color[HTML]{e6db74},
    %%%%%
    % otherkeywords={!,.,+,-,*,/,=,<,>,^,|,\&,OR,AND}
}

\lstdefinestyle{Nmonokai}{
    style=monokai,
    %%%%%
    xleftmargin=2.75em,
    framexleftmargin=2.75em,
    %%%%%
    numbers=left,
    numberstyle=\footnotesize\ttfamily\color[HTML]{f8f8f2},
    numbersep=1em
}

\lstdefinestyle{c}{
    language=C,
    style=default,
    %%%%%
    commentstyle=\color[HTML]{228B22},
    keywordstyle=\color[HTML]{0000FF},
    stringstyle=\color[HTML]{A020F0},
    emphstyle=\color[HTML]{0000FF},
    %%%%%
    emph={}
}

\lstdefinestyle{cpp}{
    language=C++,
    style=default,
    %%%%%
    commentstyle=\color[HTML]{228B22},
    keywordstyle=\color[HTML]{0000FF},
    stringstyle=\color[HTML]{A020F0},
    emphstyle=\color[HTML]{0000FF},
    %%%%%
    emph={std}
}

\lstdefinestyle{matlab}{
    language=matlab,
    style=default,
    %%%%%
    basicstyle=\footnotesize\fontfamily{pcr}\selectfont,
    numberstyle=\footnotesize\fontfamily{pcr}\selectfont,
    commentstyle=\color[HTML]{228B22},
    keywordstyle=\color[HTML]{0000FF},
    stringstyle=\color[HTML]{A020F0},
    emphstyle=\color[HTML]{0000FF},
    %%%%%
    emph={clearvars}
}

\lstdefinestyle{python}{
    language=python,
    style=default,
    %%%%%
    commentstyle=\color[RGB]{221,0,0},
    keywordstyle=[1]{\color[RGB]{255,119,0}},
    keywordstyle=[2]{\color[RGB]{144,0,144}},
    stringstyle=\color[RGB]{0,170,0},
    emphstyle=\color[RGB]{255,119,0},
    %%%%%
    emph={}
}

\lstdefinestyle{java}{
    language=java,
    style=default,
    %%%%%
    commentstyle=\color[HTML]{228B22},
    keywordstyle=\color[HTML]{0000FF},
    stringstyle=\color[HTML]{A020F0},
    emphstyle=\color[HTML]{0000FF},
    %%%%%
    emph={}
}


%%%%%%%%%%%%%%%%%%% Titlepage

\def\logopath{./resources/pdf/logo.pdf}
\def\toptitle{Université de Liège}
\title{Seconde partie du projet}
\def\subtitle{\vspace{-0.5em}INFO0009-2 -- Base de données}
%\def\authorhead{}
\author{Simon \textsc{Bernard} (s161519)\\Ivan \textsc{Klapka} (s165345)\\François \textsc{Rozet} (s161024)\\}
%\def\rightauthorhead{}
%\def\rightauthor{}
\def\context{3\up{ème} année de Bachelier Ingénieur civil}
\date{Année académique 2018-2019}

%%%%%%%%%%%%%%%%%%%

\lstdefinestyle{sql}{
    language=sql,
    style=default,
    %%%%%
    commentstyle=\color[HTML]{228B22},
    keywordstyle=\color[HTML]{0000FF},
    stringstyle=\color[HTML]{A020F0},
    emphstyle=\color[HTML]{0000FF},
    %%%%%
    emph={LOCK, UNLOCK, TABLES}
}

\lstdefinestyle{php}{
    language=php,
    style=default,
    %%%%%
    commentstyle=\color[HTML]{228B22},
    keywordstyle=\color[HTML]{0000FF},
    stringstyle=\color[HTML]{A020F0},
    emphstyle=\color[HTML]{0000FF},
    %%%%%
    emph={as, query, fetchAll, json_encode}
}

\NtbFstyle{sql}
\NtbFstyle{php}
\Nstyle{php}

%%%%%%%%%%%%%%%%%%%

\begin{document}
	\newgeometry{margin = 2.5cm}
\makeatletter
\begin{titlepage}
	\begin{minipage}[t][0.425\textheight][t]{\textwidth}
		\begin{center}
		    \ifx\toptitle\undefined
    		    \vfill
    		    \ifx\logopath\undefined
    		    \else
    			    \includegraphics[height=0.2125\textheight]{\logopath}
    			\fi
    		\else
    		    \ifx\logopath\undefined
    		    \else
    			    \includegraphics[height=0.15\textheight]{\logopath}
    			\fi
    			\vfill
    			{\huge \textsc{\toptitle}}
			\fi
			\vfill
		\end{center}
	\end{minipage}
	\vfill
	\begin{minipage}{\textwidth}
		\hspace{0.5em}
		\begin{mdframed}[linewidth = 2pt, innertopmargin = 1em, innerbottommargin = 1em, leftline = false, rightline = false]
			\begin{center}
				{\huge \bfseries \@title}
			\end{center}
		\end{mdframed}
		\hspace{0.5em}
	\end{minipage}
	\vfill
	\begin{minipage}[b][0.425\textheight][t]{\textwidth}
			\ifx\subtitle\undefined
			\else
			    \vspace{-0.5em}
			    \begin{center}
				    {\LARGE \subtitle}
				\end{center}
			\fi
			\vfill
			\ifx\rightauthor\undefined
			    \begin{center}
			        \ifx\authorhead\undefined
			        \else
		                {\large\it \authorhead\\[0.5em]}
		            \fi
			        {\large \@author}
			    \end{center}
			\else
			    \begin{minipage}[t]{0.5\textwidth}
			        \begin{flushleft}
			            \ifx\authorhead\undefined
			            \else
			                {\large\it \authorhead\\[0.5em]}
			            \fi
				        {\large \@author}
				    \end{flushleft}
				\end{minipage}
				\begin{minipage}[t]{0.5\textwidth}
				    \begin{flushright}
				        \ifx\rightauthorhead\undefined
			            \else
			                {\large\it \rightauthorhead\\[0.5em]}
			            \fi
				        {\large \rightauthor}
				    \end{flushright}
				\end{minipage}
			\fi
			\vfill
			\begin{center}
			    \ifx\context\undefined
			    \else
			        {\large \context \\[0.5em]}
			    \fi
			    {\large \@date}
			\end{center}
	\end{minipage}
\end{titlepage}
\makeatother
\restoregeometry
	\section{Arti-clever}
	\subsection{Emplacement}
	Le site web \emph{Arti-clever} est actuellement hébergé à l'adresse suivante :
	\begin{center}
	    \href{http://www.student.montefiore.ulg.ac.be/~s161024/}{http://www.student.montefiore.ulg.ac.be/~s161024/}
	\end{center}
	\subsection{Authentification}
	Pour accéder à la page d'accueil du site, et par la même occasion tous ses sous-dossiers, il est nécessaire de remplir une requête d'authentification\footnote{Les identifiants sont ceux de la base de données.} \texttt{http}. Ce sont les fichiers \texttt{.htaccess} et \text{.htpasswd} présents à la racine du site qui permettent ce type de protection. Il est à noter que l'emplacement \emph{absolu} du fichier \text{.htpasswd} est renseigné dans le \texttt{.htaccess}, ce qui signifie que pour déplacer le site, il est nécessaire de modifier ce fichier.
	\subsection{Architecture}
	Lorsque l'authentification est réussie, l'utilisateur est dirigé vers la page d'accueil du site; qui est en réalité sa seule page. Elle est séparée en plusieurs sections répondant chacune à une des sous-questions. \par
	Cette page d'accueil est un fichier \texttt{HTML} dynamisée à l'aide de \texttt{JavaScript}. Notamment, tous les accès à la base de données sont réalisés par des requêtes \texttt{AJAX} (cf. répertoire \texttt{resources/js/}) vers des scripts \texttt{PHP} et, en cas de succès, les résultats sont ajoutés dynamiquement au contenu de la page. Cela à pour avantage non négligeable de ne pas devoir recharger la page à chaque requête. \par
	Les scripts \texttt{PHP} contenus dans le répertoire \texttt{include/}, exécutent des requêtes \texttt{SQL}, pour la plupart, dépendantes des arguments fournis par méthode \texttt{get}\footnote{Nous avons préféré la méthode \texttt{get} à la méthode \texttt{post} car la première permet d'introduire des arguments dans l'URL manuellement.}. Lorsque les requêtes ne nécessitent pas d'arguments, elles sont directement rédigées dans des scripts \texttt{.sql} contenus dans le répertoire \texttt{resources/sql/}. \par
	Le site utilise la bibliothèque de styles \href{https://getbootstrap.com/}{Bootstrap}.
	\subsection{Initialisation}
    Initialiser ou réinitialiser la base de données se fait en trois étapes représentées par les trois scripts \texttt{create.sql}, \texttt{delete.sql} et \texttt{load.sql} dans le répertoire \texttt{resources/sql/ initialization/}. \par
    Le premier créée les tables si elles n'existent pas, le deuxième supprime les éventuelles données de toutes les tables et le troisième charge les données depuis les \texttt{.csv} du répertoire \texttt{resources/csv/} dans les tables correspondantes. \par
    Bien qu'il soit parfaitement envisageable de copier-coller manuellement ces scripts dans un terminal, nous avons implémenté le script \href{http://www.student.montefiore.ulg.ac.be/~s161024/include/initialize.php}{\texttt{include/initialize.php}} pour le faire à notre place. Il suffit donc de s'y rendre, pour initialiser la base de données.
    \section{Requêtes}
    Mis à part pour la question \ref{sec:nouvel-article}, les requêtes ne font que lire dans la base de données. Il n'est donc pas nécessaire de vérifier les contraintes d'intégrité.
    \renewcommand\thesubsection{\thesection.\alph{subsection}}
    \subsection{Recherche par contraintes}
    La requête est exécutée par le script \texttt{include/questions/q2a.php}.
    \lstinputlisting[style=NtbFphp, caption=\texttt{q2a.php} -- Construction de la requête \texttt{SQL}, firstnumber=18, firstline=18, lastline=33]{resources/php/q2a.php}
    Ce dernier reçoit, par l'intermidiaire du script \texttt{resources/js/questions/q2a.js}, un tableau (\texttt{\$\_GET}) dont les couples \emph{clé-valeur} définissent les contraintes sur la recherche. \par
    La valeur associée à la clé \texttt{table} est le nom de la table dans laquelle sont recherchés les éléments et les autres clés sont les noms des attributs contraints. Il est à noter que seuls sont considérés les couples dont la clé correspond à un attribut de la table. \par
    Si un attribut est un nombre, \cad{} si son \emph{type} dans la description (\texttt{DESC tablename;}) de la table contient la chaîne \texttt{int}, nous utilisons une contrainte d'égalité (\texttt{= value}) et, le cas échéant, une contrainte de contenance (\texttt{LIKE '\%value\%'}). Pour ne pas tenir compte de la casse, la commande \texttt{COLLATE UTF8\_GENERAL\_CI} est utilisée. \par
    En finalité, nous obtenons une requête du type 
    \begin{lstlisting}[style=sql, gobble=8]
        SELECT * FROM $tablename WHERE $attr0 = $value0 AND $attr1 COLLATE UTF8_GENERAL_CI LIKE '%$value1%' AND ... AND 1;
    \end{lstlisting}
    Le \texttt{1} terminal n'est normalement pas nécessaire, mais il a été ajouté pour neutraliser le dernier \texttt{AND}.
    \subsection{Ensemble de publications}
    La requête est exécutée par le script \texttt{include/questions/q2b.php}. Il est décomposé en deux parties.
    \subsubsection{Publications}
    La première partie permet d'obtenir l'ensemble de publications de l'auteur dont le matricule est \texttt{\$matricule} (initialement \texttt{\$\_GET['matricule']}).
    \lstinputlisting[style=NtbFsql, caption=\texttt{q2b.php} - Première requête \texttt{SQL}, firstnumber=16, firstline=16, lastline=30]{resources/php/q2b.php}
    La table \texttt{T1} est l'ensemble des articles dont il est auteur et la table \texttt{T2} lie à chaque article de la base de données un type (\texttt{journal} ou \texttt{conference}). Dès lors, le \texttt{NATURAL JOIN} entre ces deux tables est ce que nous souhaitons.
    \subsubsection{Seconds auteurs}
    Ensuite, pour chaque article trouvé, on rechercher ses seconds auteurs à partir de son URL. \newpage
    \lstinputlisting[style=NtbFsql, caption=\texttt{q2b.php} -- Seconde(s) requête(s) \texttt{SQL}, firstnumber=41, firstline=41, lastline=48]{resources/php/q2b.php}
    La table \texttt{T1} est donc l'ensemble des matricules des seconds auteurs de l'article dont l'URL est \texttt{\$url}. Son \texttt{NATURAL JOIN} avec la table \texttt{articles} nous permet d'obtenir leurs nom et prénom que nous concaténons en un seul champ pour faciliter l'affichage. \par
    Ensuite, le champ contenant l'URL dans la table précédente est remplacé par cette nouvelle table et ce pour chaque article.
    \subsection{Nouvel article}\label{sec:nouvel-article}
    La requête est exécutée par le script \texttt{include/questions/q2c.php}. Elle est composée de deux parties.
    \subsubsection{Pre-transaction}
    Avant d'initialiser la transaction et de verrouiller les tables pour écriture, certaines contraintes sont vérifiées (cf. lignes $60$ à $110$). \par
    Il est vérifié que
    \begin{itemize}[leftmargin=*]
        \item L'auteur existe. \Cad{} vérifier si le résultat de
        \begin{lstlisting}[style=sql, gobble=12]
            SELECT matricule FROM auteurs WHERE matricule = '$matricule_auteur';
        \end{lstlisting}
        n'est pas vide.
        \item La conférence existe. \Cad{} vérifier si le résultat de
        \begin{lstlisting}[style=sql, gobble=12]
            SELECT * FROM conferences WHERE nom = '$nom_conference' AND annee = '$annee_conference';
        \end{lstlisting}
        n'est pas vide.
        \item La journal existe. \Cad{} vérifier si le résultat de
        \begin{lstlisting}[style=sql, gobble=12]
            SELECT * FROM articles_journaux WHERE nom_revue = '$nom_revue' AND n_journal = '$n_journal';
        \end{lstlisting}
        n'est pas vide. \par
        Pour ne pas multiplier les appels à la base de données, cette requête est couplée avec une autre pour obtenir l'année de publication du journal.
        \begin{lstlisting}[style=sql, gobble=12]
            SELECT date_publication FROM articles NATURAL JOIN $type WHERE nom_revue = '$nom_revue' AND n_journal = '$n_journal' LIMIT 1;
        \end{lstlisting}
        \item L'année de la conférence ou celle du journal (cf. au dessus) est l'année du nouvel article.
        \item La page de fin d'un article de journal est au moins égale à la page de début.\footnote{Il faudrait normalement vérifier que deux articles dans un même journal ne se chevauchent pas. Néanmoins, ce n'est pas une des contraintes de l'énoncé, nous ne l'avons donc pas implémenté.}
    \end{itemize}
    \subsubsection{Transaction}
    Si la première partie est un succès, nous déclarons le début de la transaction et nous verrouillons les tables nécessaires.
    \lstinputlisting[style=NtbFphp, caption=\texttt{q2c.php} - Transaction et verrouillage des tables, firstnumber=116, firstline=116, lastline=128]{resources/php/q2c.php}
    Dans le cas d'un article de journal sans sujets mais avec des seconds auteurs, nous avons
    \begin{lstlisting}[style=sql, gobble=8]
        LOCK TABLES articles WRITE, articles_journaux WRITE, seconds_auteurs WRITE, auteurs READ;
    \end{lstlisting}
    Ensuite, nous vérifions si l'article n'existe pas. \Cad{} si
    \begin{lstlisting}[style=sql, gobble=8]
        SELECT * FROM articles WHERE url = '$url' OR doi = '$doi';
    \end{lstlisting}
    est bien vide. Cette vérification \emph{doit} être effectuée après le verrouillage pour garentir l'intégrité. \par
    Finalement, les informations liées à l'article peuvent être ajoutées aux tables. Sont choisies les requêtes nécessaires parmi les suivantes :
    \begin{description}[leftmargin=*]
        \item[articles]\mbox{}
        \begin{lstlisting}[style=sql, gobble=12]
            INSERT INTO articles (url, doi, titre, date_publication, matricule_auteur) VALUES ('$url', $doi, '$titre', '$date_publication', $matricule_auteur);
        \end{lstlisting}
        \item[articles\_conferences]\mbox{}
        \begin{lstlisting}[style=sql, gobble=12]
            INSERT INTO articles_conferences (url, presentation, nom_conference, annee_conference) VALUES ('$url', '$presentation', '$nom_conference', $annee_conference);
        \end{lstlisting}
        \item[articles\_journaux]\mbox{}
        \begin{lstlisting}[style=sql, gobble=12]
            INSERT INTO articles_journaux (url, pg_debut, pg_fin, nom_revue, n_journal) VALUES ('$url', $pg_debut, $pg_fin, '$nom_revue', $n_journal);
        \end{lstlisting}
        \item[sujets\_articles]\mbox{}
        \begin{lstlisting}[style=sql, gobble=12]
            INSERT INTO sujets_artilces (url, sujet) VALUES ('$url', '$sujet0'), ('$url', '$sujet1'), ...;
        \end{lstlisting}
        \item[seconds\_auteurs]\mbox{}
        \begin{lstlisting}[style=sql, gobble=12]
            INSERT INTO seconds_auteurs (url, matricule) VALUES ('$url', $matricule0), ('$url', $matricule1), ...;
        \end{lstlisting}
        Où les \texttt{\$matriculei} sont différents du matricule de l'auteur principal et sont ceux d'auteurs existants dans la base de données. \Cad{} que le résultat de
        \begin{lstlisting}[style=sql, gobble=12]
            SELECT matricule FROM auteurs WHERE matricule = $matriculei;
        \end{lstlisting}
        n'est pas vide pour tout \texttt{i}.
    \end{description}
    Le succès de chacune de ces requêtes est vérifié de façon à annuler l'opération en cas d'erreur.
    \begin{lstlisting}[style=php, gobble=8]
        ROLLBACK;
    \end{lstlisting}
    et la finaliser s'il n'y en a aucune.
    \begin{lstlisting}[style=sql, gobble=8]
        COMMIT;
    \end{lstlisting}
    Dans notre implémentation, ces deux commandes sont remplacées respectivement par les méthodes \texttt{rollback()} et \texttt{commit()} fournies par \texttt{PDO}. \par
    Finalement, après un succès ou un échec, nous déverrouillons les tables :
    \begin{lstlisting}[style=sql, gobble=8]
        UNLOCK TABLES;
    \end{lstlisting}
    \newpage
    \subsection{Participants actifs}
    Pour exécuter la requête, le script \texttt{resources/sql/questions/q2d.sql} est appelé.
    \lstinputlisting[style=NtbFsql, caption=\texttt{q2d.sql}]{resources/sql/q2d.sql}
    En premier lieu, nous associons (\texttt{LEFT JOIN}) à chaque ligne de la table \texttt{participations} (\texttt{T1}) les articles de l'auteur participant dans la conférence. Il est possible qu'une ligne de \texttt{T1} n'ait aucun ou plusieurs articles associés. \par
    Dans cette table, nous intéresse les matricules des participants qui ne sont \emph{pas} associés à des \texttt{NULL}. Nous réduisons donc la table aux deux colonnes de matricule et ne sélectionnons que des lignes distinctes (\texttt{DISTINCT}). \par
    Dans cette nouvelle table \texttt{T3}, tous les \texttt{matricule} figurant dans plusieurs lignes sont obligatoirement associés à une valeur \texttt{NULL}. La table \texttt{T4} groupe donc par \texttt{matricule} tout en supprimant les groupes de plus d'un élément et ceux dont la valeur associée est \texttt{NULL}. \par
    Finalement, les nom et prénom des auteurs restants sont récupérés de la table \texttt{articles}.
    \newpage
    \subsection{Sujets populaires}
    Pour exécuter la requête, le script \texttt{resources/sql/questions/q2e.sql} est appelé.
    \lstinputlisting[style=NtbFsql, caption=\texttt{q2e.sql}]{resources/sql/q2e.sql}
    Premièrement (\texttt{T1}), nous recherchons les cinq conférences les plus populaires depuis $2012$, \cad{} celles ayant eu le plus de participations. Il est à noter que si plusieurs conférences ont le même nombre de participants, le tri (\texttt{ORDER}) pourrait différer à chaque appel ainsi que les cinq premières conférences\footnote{C'est le cas de notre base de données.}. \par
    Ensuite, nous récupérons les sujets de chaque article de ces conférences et les trions par nombre d'occurrences.
\end{document}
